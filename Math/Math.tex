\section{数学相关}
\subsection{欧几里德算法}
\lstinputlisting[style=colored_cpp]{Math/Euclid.cpp}

\subsection{快速幂}
\lstinputlisting[style=colored_cpp]{Math/fast_pow.cpp}

\subsection{中国剩余定理}
\lstinputlisting[style=colored_cpp]{Math/ChineseRem.cpp}

\subsection{大数运算}
\lstinputlisting[style=colored_cpp]{Math/Bignum.cpp}

\subsection{矩阵运算}
\lstinputlisting[style=colored_cpp]{Math/Matrix.cpp}

\subsection{第二类Stirling数}
$n$个元素的集定义$k$个等价类的方法数目,记为$S(n,k)$
\begin{displaymath}
S(n, k) = \frac{1}{m!}\sum_{k=0}^{m-1}{(-1)^k C_m^k (m-k)^n}
\end{displaymath}

\begin{displaymath}
S(n, k) = 
\begin{cases}
S(n-1, k-1) + kS(n-1, k) & 2\leqslant k < n\\
1 & n=k \vee k=1 \\
0 & \text{else}
\end{cases}
\end{displaymath}
$$ $$
\lstinputlisting[style=colored_cpp]{Math/Stirling2.cpp}

\subsection{Catalan数}
$n$个节点组成的不同构二叉树个数
\begin{align*}
C_0     = & 0 \\
C_{n+1} = & \sum_{i=0}^n C_i C_{n-i}
\end{align*}
或
\begin{align*}
C_0     = & 0 \\
C_{n+1} = & \frac{2(2n+1)}{n+2} C_n
\end{align*}
\lstinputlisting[style=colored_cpp]{Math/Catalan.cpp}

\subsection{康托(Cantor)展开}
\lstinputlisting[style=colored_cpp]{Math/Cantor.cpp}

\subsection{卢卡斯(Lucas)定理}
\lstinputlisting[style=colored_cpp]{Math/Lucas.cpp}