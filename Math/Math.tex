\SetPath{Math}

\section{数学相关}

\subsection{数论算法}
\subsubsection{素数筛法}
\CPPSource{prime}
\subsubsection{欧拉函数}
欧拉函数$\phi(n)$的值为不大于n的正整数中与n互质的数的个数
\CPPSource{Euler}
\subsubsection{欧几里德(Euclid)算法}
\CPPSource{Euclid}
\RubySource{Euclid}
\subsubsection{中国剩余定理}
\CPPSource{ChineseRem}
\RubySource{ChineseRem}
\subsubsection{快速幂}
\CPPSource{fast_pow}
\subsubsection{Miller Rabin算法(素性测试)}
\CPPSource{miller_rabin}
\subsubsection{Pollard Brent算法(因数分解)}
\CPPSource{pollard_brent}

\subsection{组合数学}
\subsubsection{第二类Stirling数}
$n$个元素的集定义$k$个等价类的方法数目,记为$S(n,k)$
\begin{displaymath}
S(n, k) = \frac{1}{m!}\sum_{k=0}^{m-1}{(-1)^k C_m^k (m-k)^n}
\end{displaymath}

\begin{displaymath}
S(n, k) = 
\begin{cases}
S(n-1, k-1) + k\cdot S(n-1, k) & 2\leqslant k < n\\
1 & n=k \vee k=1 \\
0 & \text{else}
\end{cases}
\end{displaymath}
$$ $$
\CPPSource{Stirling2}

\subsubsection{Catalan数}
$n$个节点组成的不同构二叉树个数
\begin{align*}
C_0     = & 0 \\
C_{n+1} = & \sum_{i=0}^n C_i C_{n-i}
\end{align*}
或
\begin{align*}
C_0     = & 0 \\
C_{n+1} = & \frac{2(2n+1)}{n+2} C_n
\end{align*}
\CPPSource{Catalan}

\subsubsection{康托(Cantor)展开}
\CPPSource{Cantor}

\subsubsection{卢卡斯(Lucas)定理}
\CPPSource{Lucas}

\subsection{大数运算}
\CPPSource{bignum}

\subsection{矩阵运算}
\CPPSource{matrix}